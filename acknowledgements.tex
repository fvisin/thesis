\chapter*{Acknowledgments}

A life choice that spans over years rarely takes no toll. This Ph.D. has been
no exception. I am extremely grateful to all the people who kept my mental and
physical insanity bounded in these years. First of all my brother, for his wise
words when I was troubled and for all his help with moving and math, all
equally demanding. In these years I also acquired another brother, not by blood
but surely by spirit. Thank you Orhan for your incredible strength against the
slings and arrows of outrageous fortune. You helped me face losses and troubles
with a smile on my face and a beer in my hand. I couldn't have asked for more.
I am also very thankful to my parents, who raised me to be curious and
determined, and who always supported me. When I was lost and dejected, Aaron
and Cho gave me a chance and a direction. I will always be grateful for this. I
also thank Matteo for giving me the freedom to fully pursue my interest. My
gratitude goes to Adriana as well for pushing me to run more, not only
experiments. I also thank my incredible reviewers Pascal Vincent and Asja
Fisher, who did an excellent job in a short time and really helped me improve
the manuscript and my own work. There is a number of people I am thankful to,
although I cannot address them one by one without adding an appendix, it has
been a pleasure and a privilege to work with each one of them: Marco, Andrea,
Tino, Davide, Julian, Bardaripedia\textsuperscript{\textregistered}, GG,
Giulio, Ari, Ali, la zia, Beppe, Yoshua, Myriam, Pascal, Fred, Arnaud, Mathieu,
Simon, Ian, Caglar, Kyle, Vincent, Cesar, Faruk, Olivier, Simon, Laurent,
hottie, Mohammad, Michal, Mehdi, and many many more! Finally, I gratefully
acknowledge the endless patience and physical and mental support of Costanza,
who among the rest endured and encouraged my period abroad. Twice.  I also feel
a huge dept of gratitude to those who contributed to vim, git, theano, lasagne,
python, wikipedia and stackoverflow. In these times of borders and walls, open
source and shared knowledge are the best example of the virtues of cooperation.

% It is also worth mentioning Francesca Clemenza, Roberto Resmini and Dora
% Ivanof, for an organized, prompt and available secretary and technical staff
% can really affect the time a researcher devotes to his research. As well as
% their absence.%the lack of them.
