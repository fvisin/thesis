\chapter{Convolutional RNNs for Video Semantic Segmentation}\label{sec:video_segmentation}

The previous chapter introduced the task of semantic segmentation, where the
model is requested to produce a semantic mask, i.e., to classify every pixel of
an input image. Rather than presenting the algorithm with a single image, a
variant of the segmentation task, called cosegmentation, provides it with
multiple images usually taken at different locations in the same scene
(potentially by different cameras). In this setting the model can exploit the
information that the images potentially share to improve the single image
prediction and consequently improve the global score at the same time.

In many ways, \emph{video segmentation} can be seen as an extreme version of
cosegmentation, where the model is asked to segment multiple frames of a video.
As opposed to cosegmentation, where the images depict the same scene from
different angles and in moments potentially very far away in time, in videos
images come in a seamless fashion and their correlation through time can be
exploited to increase the performance.

The main problem of this task is that, if labeling images for semantic
segmentation is expensive, the time required to label all the frames of a
video is even more dramatic. One typical way to alleviate this issue is to
label only a subset of the frames, either by dropping most of the frames and
providing only a few of them with their corresponding mask, or by providing all
the input frames with only a subset of them annotated. Another way to reduce
the labeling effort is to just label the main subject of each frame or to
reduce the number of classes (to the limit case of foreground/background
separation).

Applying machine learning techniques to this class of problems is challenging
due to the lack of large amounts of labeled data, as well as the huge amount of
time required to train these kind of models; this calls for non-trivial
technological solutions. For instance, the time spent on operations that are
not strictly related to training, such as loading and preprocessing the data,
saving the weights on disk, saving samples and generating plots to monitor the
progress of the algorithm, has to be minimized. Furthermore, a recent trend in
this field is to resort to multi-GPU training; this introduces an almost linear
speedup (up to a certain numbers of GPUs)~\citep{theano2016short,ma2016theano},
but increases the complexity of the algorithm, is often subject to constraints
(e.g., the GPUs might have to be hosted by the same machine or node of a
cluster), and makes debugging much less cumbersome in many cases.

\autoref{sec:reseg} introduced ReSeg~\citep{Visin_2016_CVPR_Workshops}, an
RNN-based model for image semantic segmentation. One of the main strengths of
ReSeg comes from the coupling of fast general-purpose CNN feature extractors
with stateful RNNs able to carry the information extracted by the CNNs through
various steps of computation, showing the effectiveness of RNNs applied to the
spatial domain. This approach can be taken a step further, allowing the RNNs
to jointly process space and time information in videos by placing the CNNs
directly into the inner computation of the recursive layer. This chapter
introduces a novel architecture that aims to overcome the task separation
traditionally enforced by the CNN-RNN dichotomy. This is achieved by processing
the temporal and the spatial information jointly, mixing the two architectures
at each layer of the recurrent-convolutional hierarchy.


\section{Motivation}

% Intro
In the last few years, convolutional neural networks have been
successfully applied to address many computer vision tasks, such as image
classification~\citep{Krizhevsky2012-alexnet,Simonyan2015,
Szegedy-et-al-arxiv2014} and object detection~\citep{Girshick-et-al-arxiv2013,
Sermanet13overfeat} and their application has become ubiquitous in many widely
used commercial products, such as face and smile detectors in digital cameras,
people recognition in social networks and street signals recognition in cars.

Recently, pixel-wise prediction of still images has enjoyed the attention
of the computer vision community. In contrast to image classification and
object detection this \emph{structured prediction} problem requires each
pixel to be classified, which demands for a more detailed understanding of
the image, as well as consistency in the prediction. In this domain, many
state of the art architectures successfully coupled classification models
trained on ImageNet~\citep{Simonyan14vgg,Szegedy15googlelenet} with various
upsampling strategies, to address the challenging image segmentation and scene
parsing task~\citep[see e.g.,~][]{long2014fully,noh2015learning} in a trainable
end-to-end fashion.

The success witnessed on still images has rapidly reached the video domain,
achieving remarkable results in tasks such as video action recognition~\citep{
simonyan2014two,karpathy2014large}, event detection in videos~\citep{
yeung2015end} and video captioning~\citep{yao2015describing}. Along this
direction, some effort has been devoted lately to address the more
challenging task of end to end video semantic segmentation~\citep{Tran16v2v}.

While many computer vision video-related tasks involve predicting only a single
or a few output per video, some of them -- such as video semantic
segmentation, change detection, and object tracking -- require a
\emph{per temporal "voxel"} prediction. As an analogy to 3D voxels, where a
volumetric pixel (or voxel) is used to describe a local portion of the data
that extends across space in three dimensions, temporal voxels refer to 3D
portions of the data that span over a contiguous two-dimensional space and over
a one-dimensional sequence in the temporal domain.
Works in this direction usually define a fixed \emph{a priori} window on the
temporal dimension to obtain a voxel-wise prediction~\citep{Tran16v2v}.

The most common way to address video related tasks with neural network
approaches is to combine CNNs with RNNs in a sequential fashion, by first
applying a (possibly pre-trained) CNN on each frame and then interpreting the
resulting feature maps in a temporal consistent way by processing them with an
RNN~\citep{Donahue-et-al-arxiv2014,Vinyals-et-al-CVPR2015,Karpathy+Li-CVPR2015,
Venugopalan_2015_ICCV}. While this is an effective way to exploit the temporal
information, the two pipelines do not completely benefit from each other, since
the temporal consistency is enforced only in the last step. Furthermore, the
spatial relations are mostly ignored by the state of the RNN as they are taken
into account only through the encoding coming from the CNNs.

On this note, \cite{ShiCWYWW15} propose an interesting approach that deeply
entangles the convolution operator with the state-to-state and input-to-state
transitions of the RNN itself, rather than encoding each frame independently
with CNNs and then processing the spatial encodings timewise with an RNN. The
model presented in this chapter, called Decoder Encoder Convolutional LSTM
(DEConvLSTM), builds on top of~\cite{ShiCWYWW15} and extends their work in two
ways: i) replacing the convolution operator in input-to-state transitions with
a full multi-layer convolutional subnetwork ii) introducing a novel
recurrent-convolutional-upsampling layer that learns how to leverage the
spatio-temporal dimensions jointly to produce an upsampled feature map that
respects the 2D topology of the input feature map and at the same time
maintains its time consistency. The full model mimics the conventional 2D
semantic segmentation encoder-decoder architecture, but is able to jointly
capture spatio-temporal information, using the internal memory of RNNs instead
of a fixed window over time.

% We demonstrate the effectiveness of recurrent-convolutional LSTM for video
% semantic segmentation task.
% We provide and in-depth analysis on the importance of the temporal dimension
% for this kind of task.
% - how temporal dimension is important for the task


\section{Model description}\label{sec:deconvlstm_model}

Videos can be interpreted as a set of frames that interlace spatial information
through time. The supporting idea of the proposed DEConvLSTM model builds on
this, with the claim that space and time cannot be processed sequentially -- as
done by much of the past ML literature -- but rather the two should be mixed
together in a coherent and joint architecture. The DEConvLSTM intertwines space
with time at each step of the hierarchy~(see~\autoref{fig:deconvlstm_model}),
taking inspiration from the convolutional LSTM (ConvLSTM) model described
by~\citep{ShiCWYWW15}.
% and pushing that intuition further.
As many successful models for semantic segmentation on static images, the
DEConvLSTM couples an encoding and a decoding pathway, that respectively shrink
the resolution of the intermediate feature maps, encoding a rich representation
of the input, and upsample it to recover the resolution of the input video.

This section will introduce the novelties of the DEConvLSTM model, starting
with a description of the encoding pathway that builds on the ConvLSTM model
and then building on that to uncover the novelties of the proposed model.

\begin{figure}[t]
    \centering
    \includegraphics[width=0.8\columnwidth]{img/deconvLSTM/DEConvLSTM.png}
    \caption{The DEConvLSTM architecture. The input (bottom) to the model is a
        sequence of video frames and the output (top) is a sequence of
        segmentation maps. The rectangles represent ConvLSTM and TransConvLSTM
        layers. The inner convolutions of these layers are not represented for
        space reasons. The black solid arrows represent connections between
        layers in the hierarchy, the dashed arrows represent the recursive
        connections in the time domain. Finally the yellow arrows are the skip
        connections.}\label{fig:deconvlstm_model}
\end{figure}


\subsection{Convolutional LSTM}

In~\citep{ShiCWYWW15} the authors propose the ConvLSTM model that is a
modification of the well-known LSTM with peephole connections. To allow for a
more convenient comparison, the equations of the vanilla LSTM with peephole
connections (that were defined in~\autoref{eq:LSTM_peepholes}) are reported
here

\begin{equation*}\tag{\ref{eq:LSTM_peepholes} revisited}
\begin{split}
    \mathbf{i}_t &= \sigma\left(
        \mathbf{W}_{ih} \cdot \mathbf{h}_{t-1} +
        \mathbf{W}_{ix} \cdot \mathbf{x}_t +
        \mathbf{W}_{ic} \hadamard \mathbf{c}_{t-1} +
        \mathbf{b}_i \right),\\
    \mathbf{f}_t &= \sigma\left(
        \mathbf{W}_{fh} \cdot \mathbf{h}_{t-1} +
        \mathbf{W}_{fx} \cdot \mathbf{x}_t +
        \mathbf{W}_{fc} \hadamard \mathbf{c}_{t-1} +
        \mathbf{b}_f \right),\\
    \mathbf{\tilde c}_t &= tanh \left(
        \mathbf{W}_{ch} \cdot \mathbf{h}_{t-1} +
        \mathbf{W}_{cx} \cdot \mathbf{x}_t +
        \mathbf{b}_c \right),\\
    \mathbf{c}_t &= \mathbf{f}_t \hadamard \mathbf{c}_{t-1} + \mathbf{i}_t
        \hadamard \mathbf{\tilde c}_t,\\
    \mathbf{o}_t &= \sigma\left(
        \mathbf{W}_{oh} \cdot \mathbf{h}_{t-1} +
        \mathbf{W}_{ox} \cdot \mathbf{x}_t +
        \mathbf{W}_{oc} \hadamard \mathbf{c}_{t-1} +
        \mathbf{b}_o \right),\\
    \mathbf{h}_t &= \mathbf{o}_t \hadamard tanh \left(\mathbf{c}_t\right),
\end{split}
\end{equation*}

\noindent where $\mathbf{x}_t$ and $\mathbf{h}_t$ are the input and output
feature maps respectively at time $t$; $\mathbf{c}_t$ is the state of the cell
(or memory) of the recurrent unit at time $t$; $\mathbf{i}_t$, $\mathbf{f}_t$,
$\mathbf{o}_t$ are respectively the states of the input gate, the forget gate,
and the output gate, i.e., the gates that control the behavior of the recurrent
unit given the input and its state at each time step $t$. $\sigma$ is the
logistic function, $\mathbf{W}_{\star}$\footnote{
    The $\star$ notation here is used as a placeholder to represent any letter
    or string. In this particular case it takes the place of \textit{i, f, c}
    and \textit{o}.},
the weights and $\hadamard$ the Hadamard product.

The convolutional LSTM of~\citep{ShiCWYWW15} replaces the dot products with
convolutions, leading to the following formulation

\begin{equation}
\begin{split}
    \mathbf{i}_t &= \sigma\left(
        \mathbf{W}_{ih} * \mathbf{h}_{t-1} +
        \mathbf{W}_{ix} * \mathbf{x}_t +
        \mathbf{W}_{ic} \hadamard \mathbf{c}_{t-1} +
        \mathbf{b}_i \right),\\
    \mathbf{f}_t &= \sigma\left(
        \mathbf{W}_{fh} * \mathbf{h}_{t-1} +
        \mathbf{W}_{fx} * \mathbf{x}_t +
        \mathbf{W}_{fc} \hadamard \mathbf{c}_{t-1} +
        \mathbf{b}_f \right),\\
    \mathbf{\tilde c}_t &= tanh \left(
        \mathbf{W}_{ch} * \mathbf{h}_{t-1} +
        \mathbf{W}_{cx} * \mathbf{x}_t +
        \mathbf{b}_c \right),\\
    \mathbf{c}_t &= \mathbf{f}_t \hadamard \mathbf{c}_{t-1} + \mathbf{i}_t
        \hadamard \mathbf{\tilde c}_t,\\
    \mathbf{o}_t &= \sigma\left(
        \mathbf{W}_{oh} * \mathbf{h}_{t-1} +
        \mathbf{W}_{ox} * \mathbf{x}_t +
        \mathbf{W}_{oc} \hadamard \mathbf{c}_{t-1} +
        \mathbf{b}_o \right),\\
    \mathbf{h}_t &= \mathbf{o}_t \hadamard tanh \left(\mathbf{c}_t\right),
\end{split}
\end{equation}

\noindent where $*$ is the convolution operator. Notably, as opposed to
classical recurrent architectures, all gates and states here have \emph{spatial
dimensions} as they are the result of the application of convolutions on the
input of the layer.

The encoding pathway of our DeconvLSTM model builds on this, with two
modifications: i) it introduces a full, multi-layered, CNN, in place of the
convolution operator; ii) it replaces the Hadamard product with a convolution.
This is expressed in~\autoref{eq:deconvLSTM_enc}
%resulting in the following equations

\begin{equation}\label{eq:deconvLSTM_enc}
\begin{split}
    \mathbf{i}_t &= \sigma\left(
        \mathbf{W}_{ix}^{(L)} * \dots * \rho \mathbf{W}_{ix}^{(1)} *
            \rho \mathbf{W}_{ix}^{(0)} * \mathbf{x}_t +
        \mathbf{W}_{ih} * \mathbf{h}_{t-1} +
        \mathbf{W}_{ic} * \mathbf{c}_{t-1} +
        \mathbf{b}_i \right),\\
    \mathbf{f}_t &= \sigma\left(
        \mathbf{W}_{fx}^{(L)} * \dots * \rho \mathbf{W}_{fx}^{(1)} *
            \rho \mathbf{W}_{fx}^{(0)} * \mathbf{x}_t +
        \mathbf{W}_{fh} * \mathbf{h}_{t-1} +
        \mathbf{W}_{fc} * \mathbf{c}_{t-1} +
        \mathbf{b}_f \right),\\
    \mathbf{\tilde c}_t &= tanh \left(\mathbf{W}_{cx}^{(L)} * \dots *
            \rho \mathbf{W}_{cx}^{(1)} * \rho \mathbf{W}_{cx}^{(0)} *
            \mathbf{x}_t +
        \mathbf{W}_{ch} * \mathbf{h}_{t-1} + \mathbf{b}_c \right),\\
    \mathbf{c}_t &= \mathbf{f}_t \hadamard \mathbf{c}_{t-1} + \mathbf{i}_t
        \hadamard \mathbf{\tilde c}_t,\\
    \mathbf{o}_t &= \sigma\left(
        \mathbf{W}_{ox}^{(L)} * \dots * \rho \mathbf{W}_{ox}^{(1)} *
            \rho \mathbf{W}_{ox}^{(0)} * \mathbf{x}_t +
        \mathbf{W}_{oh} * \mathbf{h}_{t-1} +
        \mathbf{W}_{oc} * \mathbf{c}_{t-1} +
        \mathbf{b}_o \right),\\
    \mathbf{h}_t &= \mathbf{o}_t \hadamard tanh \left(\mathbf{c}_t\right).
\end{split}
\end{equation}

\noindent A $ReLU$ nonlinearity $\rho$ follows every convolution on the input
$\mathbf{W}_{\star x}^{(l \in {1,\dots,L})}$ apart from the last ones
$\mathbf{W}_{\star x}^{(L)}$, since the usual $\sigma$ nonlinearity is already
applied to the output of the layer to keep the typical LSTM behavior intact.
The last convolutions $\mathbf{W}_\star^{(L)}$ are also followed by a pooling
layer (not expressed in the equations) to reduce the dimensionality of the
intermediate feature maps before the nonlinearity is applied.

The substitution of the Hadamard products with convolutions reduces the number
of parameters and is consistent with the idea of retaining the spatial
structure in the internal state and gates of the LSTM. It also allows the
network to process videos with a different resolution than the one it was
trained on, which is often a convenient property.


\subsection{Transposed Convolutional LSTM}

It has become common to address the problem of segmenting still images with
architectures that first shrink the resolution of the feature maps to produce a
lower-dimensionality yet rich encoding, and then upsample the intermediate
feature maps back to the original image size in order to emit a per-pixel
prediction~(see~\autoref{sec:reseg}). This kind of structure has proven to be
very effective to segment 2D images.

Transferring this knowledge to the video segmentation task, the DEConvLSTM
model introduces a novel Transposed Convolutional LSTM (TransConvLSTM) layer
that exploits transposed convolutions to recover the resolution lost due to the
pooling layers. In the same fashion as for the ConvLSTM layer, the
TransConvLSTM layer inserts a transposed CNN~\emph{inside} the state update
function of an LSTM with peephole connections

\begin{equation}\label{eq:deconvLSTM_dec}
\begin{split}
    \mathbf{i}_t &= \sigma\left(
        \mathbf{W}_{ix}^{(L)} \circledast \dots \circledast \rho
            \mathbf{W}_{ix}^{(1)} \circledast \rho \mathbf{W}_{ix}^{(0)}
            \circledast \mathbf{x}_t +
        \mathbf{W}_{ih} * \mathbf{h}_{t-1} +
        \mathbf{W}_{ic} * \mathbf{c}_{t-1} +
        \mathbf{b}_i \right),\\
    \mathbf{f}_t &= \sigma\left(
        \mathbf{W}_{fx}^{(L)} \circledast \dots \circledast \rho
            \mathbf{W}_{fx}^{(1)} \circledast \rho \mathbf{W}_{fx}^{(0)}
            \circledast \mathbf{x}_t +
        \mathbf{W}_{fh} * \mathbf{h}_{t-1} +
        \mathbf{W}_{fc} * \mathbf{c}_{t-1} +
        \mathbf{b}_f \right),\\
    \mathbf{\tilde c}_t &= tanh \left(
        \mathbf{W}_{cx}^{(L)} \circledast \dots \circledast \rho
            \mathbf{W}_{cx}^{(1)} \circledast \rho \mathbf{W}_{cx}^{(0)}
            \circledast \mathbf{x}_t +
        \mathbf{W}_{ch} * \mathbf{h}_{t-1} +
        \mathbf{b}_c \right),\\
    \mathbf{c}_t &= \mathbf{f}_t \hadamard \mathbf{c}_{t-1} + \mathbf{i}_t
        \hadamard \mathbf{\tilde c}_t,\\
    \mathbf{o}_t &= \sigma\left(
        \mathbf{W}_{ox}^{(L)} \circledast \dots \circledast \rho
            \mathbf{W}_{ox}^{(1)} \circledast \rho \mathbf{W}_{ox}^{(0)}
            \circledast \mathbf{x}_t +
        \mathbf{W}_{oh} * \mathbf{h}_{t-1} +
        \mathbf{W}_{oc} * \mathbf{c}_{t-1} +
        \mathbf{b}_o \right),\\
    \mathbf{h}_t &= \mathbf{o}_t \hadamard tanh \left(\mathbf{c}_t\right).
\end{split}
\end{equation}

\noindent The TransConvLSTM applies multiple layers of transposed convolutions
($\circledast$) with unit stride and "same" padding, followed by a last
transposed convolution whose stride is a hyperparameter of the model and whose
output shape is fixed to be the same as the one of the corresponding ConvLSTM
in the encoding pathway (when stride $> 1$ multiple output shapes are possible,
see the $a$ parameter in~\autoref{rel:padding_strides_transposed_odd}
in~\autoref{sec:background}). Finally, as was done for the encoding pathway,
every convolution on the input apart from the last one is followed by a $ReLU$
nonlinearity to keep the typical LSTM behavior intact.

\subsection{Bidirectional RNNs}\label{deconvLSTM_biRNN}
Bidirectional RNNs (Bi-RNNs)~\citep{Schuster1997bidirecrnn} have been
successfully employed in the literature for image and video related
tasks~\citep{Graves+Schmidhuber-2009,visin2015renet,Du2015_CVPR,
Visin_2016_CVPR_Workshops}. Given an input sequence, Bi-RNNs move over it in
both directions at the same time, i.e., from the first element forwards and
from the last element backwards. This allows them to integrate information
coming from previous and following elements at each position of the sequence.
In videos, this ensures a smoother temporal consistency in the predictions. In
contrast with~\citep{ShiCWYWW15}, the DEConvLSTM model exploits bidirectional
LSTMs to enforce temporal coherence in the predictions.

The forward and backward passes can be combined at each time step either via
concatenation over the feature maps axis, as initially suggested by~\citep{
Schuster1997bidirecrnn}, or by summing them elementwise. Both alternatives have
been explored in the early exploratory experiments for the DEConvLSTM model,
including a variant of the sum that learns the parameters of a weighted sum
rather than performing an arithmetic mean

\begin{equation}
    RNN_t^{\leftrightarrow} = \alpha_t RNN_t^{\leftarrow} +
        (1 - \alpha_t) RNN_t^{\rightarrow}, \quad\text{for $\forall t$ in
        sequence},
\end{equation}

\noindent and an alternative variant with the same parameter $\alpha$ shared
across all the time steps. Concatenation seems to give slightly better
performance and has therefore been selected as combination method for the
rest of the experiments. However, as a future work, it could be interesting to
assess whether the accuracy gain justifies the higher memory consumption.
% Due to memory constraints, we don't follow the trend of concatenating the B-RNN
% outputs \citep{}, but rather sum them


\subsection{Skip Connections}\label{sec:deconvLSTM_skip_connections}
One very popular component of many recent ANN-based architectures are skip
connections~\citep[see e.g.,~][]{sermanet-cvpr-13,liu2015parsenet,
long2015fully,Srivastava-et-al-arxiv2015,bell2015inside,he2015deep,
hariharan2015hypercolumns,honari2016recombinator}, which add a connection
between earlier layers in the hierarchy and downstream layer~\emph{skipping}
the intermediate ones. This has the desirable property of ensuring a gradient
flow to the early layers of the architecture even in very deep models and has
been proven to improve the performance of many tasks. Furthermore, skip
connections are especially useful when high resolution spatial information is
needed in the last stages. When it comes to images and videos semantic
segmentation this is a crucial element to recover detailed informations on
where the objects are, and this improves the upsampling precision of the
decoding pathway.

The DEConvLSTM model follows this well-established practice as well. The
TransConvLSTM layers are parametrized so that they emit a feature map of the
same resolution as the corresponding encoding layer~(see~\autoref{%
fig:deconvlstm_model}). The skip connections reflect this shape pattern linking
the encoding layers to the decoding layers with the same resolution. Similarly
to the case of bidirectional RNNs~(see~\autoref{deconvLSTM_biRNN}),
concatenating the skipped feature maps with the corresponding upsampled feature
maps rather than summing them together seems to give the best results. The
reason for this might be that the elementwise sum of the feature maps
constrains the model to be conservative in the upsampling pathway not to
overcome the information coming from the early layers. On the contrary, by
concatenating the feature maps together the model has complete freedom on the
values and the magnitude of the upsampled feature maps and the information
coming from the skipped connections is always preserved.


\section{Experiments}\label{sec:deconvLSTM_experiments}
% \subsection{FCN8}
% FCN8 network achieved state of the art results on semantic segmentation tasks
% \citep{DBLP:conf/cvpr/LongSD15}. The FCN8 network is divided into two parts : a
% contracting part and a decontracting part, so that the output of the network
% has the same size than the input. The contracting part is adapted from vgg
% \citep{Simonyan14vgg}. The decontracting part is made of 3 transposed
% convolutions, to what paper meaning, add citation, which output a slighty
% larger map than the input. A crop layer then reshape the output to the exact
% same size of the input. Two skip connections are added between layers in the
% contracting path and the corresponding one in decontracting path.
%
% We reused the weights of the contrating path without retraining them. However,
% we retrain all the decontracting path when used.
%
% This FCN8 network is used a baseline in 3 configurations :
% \begin{itemize}
%     \item As is. No modification to the network is done. Only the decontracting
%         path is trained. The network has no temporality taken into account.
%     \item We add a convolutional LSTM layer on top of the network. The time is
%         taken into account only at the top scale.
%     \item The transposed convolutions of the decontracting path are replaced by
%         upsampling LSTM layers. The temporality is taken into account at
%         different scales in the decontracting path. Like in the previous point
%         convolutional LSTM layer is added on top. The skip connections of the
%         original FCN8 are removed.
% \end{itemize}

\subsection{Datasets}\label{sec:deconvLSTM_datasets}

\begin{figure}[t]
    \centering
    \includegraphics[width=0.8\columnwidth]{img/deconvLSTM/camvid_gt.jpg}
    \caption{An example of ground truth segmentation for the CamVid dataset.}
    \label{fig:deconvlstm_camvid_gt}
\end{figure}

\begin{figure}[t]
    \centering
    \includegraphics[width=\columnwidth]{img/deconvLSTM/gatech_gt.png}
    \caption{An example of ground truth segmentation for the Gatech dataset.}
    \label{fig:deconvLSTM_gatech_gt}
\end{figure}

\begin{figure}[t]
    \centering
    \includegraphics[width=0.3\columnwidth]{img/deconvLSTM/davis_gt1.jpg}
    \includegraphics[width=0.3\columnwidth]{img/deconvLSTM/davis_gt2.jpg}
    \includegraphics[width=0.3\columnwidth]{img/deconvLSTM/davis_gt3.jpg}\\
    \includegraphics[width=0.3\columnwidth]{img/deconvLSTM/davis_gt4.jpg}
    \includegraphics[width=0.3\columnwidth]{img/deconvLSTM/davis_gt5.jpg}
    \includegraphics[width=0.3\columnwidth]{img/deconvLSTM/davis_gt6.jpg}\\
    \includegraphics[width=0.3\columnwidth]{img/deconvLSTM/davis_gt7.jpg}
    \includegraphics[width=0.3\columnwidth]{img/deconvLSTM/davis_gt8.jpg}
    \includegraphics[width=0.3\columnwidth]{img/deconvLSTM/davis_gt9.jpg}\\
    \caption{An example of ground truth segmentation for the DAVIS dataset.}
    \label{fig:deconvlstm_davis_gt}
\end{figure}


\subsubsection{CamVid: Motion-based Video Segmentation}
\label{sec:deconvLSTM_camvid}
The Cambridge-driving Labeled Video Database (CamVid)~\citep{
Brostow2010semantic}~\footnote{%
http://mi.eng.cam.ac.uk/research/projects/VideoRec/CamVid/}
has been introduced in~\autoref{sec:reseg+camvid}. It is a motion based video
segmentation dataset composed by three sequences captured from the perspective
of a person driving a vehicle, with resolution $960 \times 720$ pixels, at 30
frames per second. The dataset contains 32 classes and a small number of pixels
have been labeled as "void", when it was not possible to find a proper match
with any of the 32 classes. The CamVid dataset consists of 367 training, 101
validation, and 233 test images, for a total of 701 images and corresponding
annotations The DEConvLSTM model has been trained on the same variant of the
dataset as the one used by~\citep{badrinarayanan2015segnet}, that consists of a
subset of 11 class categories and downsamples the images by a factor of 2
resulting in a final $480 \times 360$ resolution. The dataset contains video
sequences, but in the literature it has mostly been used for still image
semantic segmentation. See~\autoref{fig:deconvlstm_camvid_gt} for a sample with
ground truth from each of the three sequences.


\subsubsection{Gatech: Geometric Context Video Segmentation}
\label{sec:deconvLSTM_gatech}
The Gatech dataset~\citep{VideoGeometricContext2013}~\footnote{%
http://www.cc.gatech.edu/cpl/projects/videogeometriccontext/}
is a geometric scene understanding dataset, which consists of 63 videos for
training/validation and 38 for testing. Each video has between 50 and 300
frames (with an average of 190). A pixel-wise segmentation map is provided for
each frame. There are 8 classes in the dataset: \textit{sky}, \textit{ground},
\textit{buildings}, \textit{porous}, \textit{humans}, \textit{cars},
\textit{vertical mix} and \textit{general mix}. The dataset was originally built
to learn 3D geometric structure of outdoor video scenes. A sample with ground
truth segmentation taken from the Gatech dataset can be seen
in~\autoref{fig:deconvLSTM_gatech_gt}.


\subsubsection{DAVIS: Densely Annotated Video Segmentation}
\label{sec:deconvLSTM_davis}
The DAVIS dataset~\citep{Perazzi2016}~\footnote{
https://graphics.ethz.ch/~perazzif/davis/} has been made publicly available
very recently. It contains 50 high-resolution Full HD video sequences, with
all the frames densely labeled. The sequences have been carefully chosen to
contain the typical major challenges of video segmentation and the dataset
comes with a comprehensive benchmark of many state-of-the-art methods on
three complementary metrics that measure the spatial accuracy of the
segmentation, the quality of the silhouette and the temporal coherence. Due to
memory restrictions, these tests were run on a version of the dataset
downsampled at 480p resolution. To make it possible to compare with this
state-of-the-art, the DEConvLSTM has been trained on the resized version too.
Finally, the videos are annotated with attributes such as \textit{occlusion},
\textit{non-linear deformation} and \textit{motion blur}. These annotations
were not taken into account when training the DEConvLSTM model on this dataset.
A ground truth segmentation example for the DAVIS dataset can be seen
in~\autoref{fig:deconvlstm_davis_gt}.

% \subsubsection{Change Detection}
% \label{sssec:changedet}
% Change Detection dataset (CDnet) comes after 2 successful benchmarks
% \citep{wang2014cdnet,goyette2012changedetection}. CDnet contains over 160,000
% frames obtained from 31 videos and manually annotated. This dataset covers a
% variety of semantic segmentation scenarios. The categories include Challenging
% weather, Air turbulence, camera jitter, dynamic background, low resolution
% traffic videos, intermittent object motion, shadow, thermal, bad weather, low
% frame rate, night videos, pan-tilt-zoom and turbulence. The official evaluation
% metrics are Recall (Re), Specificity (Sp), False Positive Rate (FPR), False
% Negative Rate (FNR), Percentage of Wrong Classifications (PWC), Precision (Pr)
% and F-Measure (FM).


\subsection{Data augmentation and preprocessing}
%it has affine transformation, flipping, cropping, warping, etc...
Training on video semantic segmentation datasets is very time consuming, in
fact the prediction of a segmentation mask requires to densely process a
temporal window for each frame. This limits the size of the batches that can
be processed and slows down training.

With the goal of collecting a preliminary assessment of the performance of the
model and of its flexibility in tackling different kinds of datasets, we chose
to keep the training time as contained as possible without resorting to data
augmentation techniques. However, it has to be pointed out that some degree of
improvement can be expected from the introduction of some data augmentation,
e.g., by distorting the frames with affine transformations, by flipping and
rotating them or by shifting the intensity values of all the pixels by a random
value.

In the same spirit, the only preprocessing technique adopted -- apart from
resizing the videos to allow a comparison with the state-of-the-art
(see~\autoref{sec:deconvLSTM_datasets}) -- was to crop the frames randomly
selecting an image patch of $(224, 224)$ resolution during training to reduce
the memory usage and speed up the training.

\subsection{Experimental settings}\label{sec:deconvLSTM_settings}
% In this section, we study the effectiveness of involving the temporal
% information on the task of video spatial segmentation. Our experiments with
% regarding to our pipeline are designed to find an optimum way of involving the
% temporal information.
% Toward this end, we compare our end to end segmentation method with two
% following baselines:
% \begin{itemize}
%     \item Segmentation over each frame independently using CNNs. In this
%         case, we don't involve any temporal information.
%     \item Employing the temporal information only on half of the network. (TODO:
%         why are we using this as a baseline?)
% \end{itemize}
% As explained earlier we employ the temporal information on every layers of the
% network and have a recursion over time at each layer. Therefore, we have both
% convolutional LSTM and upsampling LSTM. (TODO: requires more explanation)

% \subsubsection{Configuration of the layers}
% The convolutional and upsampling layers respectively consist of 70 and 100
% units. We apply batch normalization on each scale.
A first exploratory phase allowed to test several different architectures on
the relatively small CamVid dataset in order to select a suitable starting
point for the rest of the experiments. The principal hyperparameters of the
DEConvLSTM model are the number of encoding and decoding layers, as well as the
depth of their inner convolutions. As explained in~\autoref{%
sec:deconvLSTM_skip_connections}, we chose to constrain, by design, the number
of decoding layers to be the same as the number of encoding layers, to allow
for the skip connections structure to be symmetrical with respect to the middle
feature map resolution of the network. The number of encoding (and therefore
decoding) layers that empirically appeared to be optimal is 5, and has been
used consistently throughout the experiments.  Similarly, the number of inner
convolutions and transposed convolutions has been set to 2 (per layer) and
their hyperparameters -- namely the number of filters per
sublayer~(\textbf{nf}), filters size per sublayer~(\textbf{fs}) and stride per
sublayer~(\textbf{st}) -- have been fixed as shown in~\autoref{%
tbl:deconvLSTM_enc_hyperparams} and \autoref{tbl:deconvLSTM_dec_hyperparams}.

\begin{table}[t]
    \centering{
        \small
        \begin{tabular}{c|c|c|c|c|c}
            \textbf{Layers} & 1 & 2 & 3 & 4 & 5\\ \hline
            \textbf{nf} & [64, 64] & [128, 128] & [256, 256] & [384, 384] &
                [512, 512]\\
            \textbf{fs} & [1x1, 3x3] & [1x1, 3x3] & [1x1, 3x3] & [1x1, 3x3] &
                [1x1, 3x3]\\
            \textbf{st} & [2, 2] & [2, 2] & [2, 2] & [2, 2] & [2, 2]\\
        \end{tabular}
    }
    \caption{ConvLSTM hyperparameters. Number of filters per
        sublayer~(\textbf{nf}), filters size per sublayer~(\textbf{fs}) and
        stride per sublayer~(\textbf{st}).}
    \label{tbl:deconvLSTM_enc_hyperparams}
\end{table}

\begin{table}[t]
    \centering{
        \small
        \begin{tabular}{c|c|c|c|c|c}
            \textbf{Layers} & 1 & 2 & 3 & 4 & 5\\ \hline
            \textbf{nf} & [384, 384] & [256, 256] & [128, 128] & [64, 64] &
                [64, 64]\\
            \textbf{fs} & [1x1, 3x3] & [1x1, 3x3] & [1x1, 3x3] & [1x1, 3x3] &
                [1x1, 3x3]\\
            \textbf{st} & [2, 2] & [2, 2] & [2, 2] & [2, 2] & [2, 2]\\
        \end{tabular}
    }
    \caption{TransConvLSTM hyperparameters. Number of filters per
        sublayer~(\textbf{nf}), filters size per sublayer~(\textbf{fs}) and
        stride per sublayer~(\textbf{st}).}
    \label{tbl:deconvLSTM_dec_hyperparams}
\end{table}

Throughout the experiments a light Dropout~\citep{Srivastava14} with drop
probability $0.2$ was used as regularization technique, as well as a
regularization of magnitude $0.0001$ for both the neural network weights and
the batch normalization. Optimization was carried out with
RMSProp~\citep{tieleman2012lecture} with learning rate $1e^{-4}$ and gradient
clipping of $10$. Finally, feature-wise batch normalization was applied before
each layer with an amount of momentum that varied from dataset to dataset.


\subsection{Results}\label{sec:deconvLSTM_results}

The DEConvLSTM model has been studied on three different datasets. A first
set of experiments was run on CamVid to fix some of the hyperparameters of the
model. Afterwards, the bigger Gatech and DAVIS datasets were taken into
account.

%%%% CAMVID %%%%%
All the experiments on CamVid were run with batch size $5$, training on cropped
patches of $224 \times 224$. The learning procedure was interrupted with early
stopping on the average Intersection over Union (IoU) of the validation set.
The main goal of these experiments was not to reach state-of-the-art
performance, but rather to assess the sensitivity of the model to different
hyperparameters. \autoref{tbl:deconvLSTM_camvid_overfit} reports the results of
some of these preliminary experiments with different optimizers and various
sequence lengths. The models suffer from some generalization issues on this
dataset. This is not too surprising given the limited number of images in the
CamVid dataset, as will be discussed in more detail
in~\autoref{sec:deconvLSTM_discussion}.

The results of the DEConvLSTM model with sequences of length $3$, fine-tuned
for 100 epochs on images at full resolution is compared
in~\autoref{tbl:deconvLSTM_camvid} with the state-of-the-art on CamVid. The
model performs comparably to the state-of-the-art of the last year; there is a
margin for improvement, which could be probably gained, at least in part, by
reducing the generalization issues with stronger regularization and relying on
data augmentation to diminish the overfitting. However not too much effort was
spent on hyperparameters tuning on this dataset, as the main goal was to gain
confidence with the model rather than to reach the state-of-the-art.

\begin{table}[t]
    \centering{\small
        \begin{tabular}{l|c||c}
            Method & Global accuracy & Avg IoU\\ \hline

            SegNet-Basic \citep{badrinarayanan2015segnet} & 82.8 & 46.3 \\
            SegNet \citep{badrinarayanan2015segnet} &  88.6 & 50.2 \\
            ReSeg \citep{Visin_2016_CVPR_Workshops} & 88.7 & 58.8 \\
            Dilation \citep{yu2015multi} & n/a & 65.3 \\
            \textbf{Dilation + FSO - DiscreteFlow} \citep{kundu2016feature} & n/a & \textbf{66.1} \\
            \emph{DEConvLSTM (finetuned)} & 87.0 & 53.9
        \end{tabular}
    }
    \caption{Results on the CamVid dataset. Pixel accuracy and average
        Intersection over Union (IoU) are reported. Higher is better.}
    \label{tbl:deconvLSTM_camvid}
\end{table}

\begin{table}[t]
    \centering{\small
        \begin{tabular}{c|c||c|c|c||c|c|c}
            & & \multicolumn{3}{c||}{Global accuracy} & \multicolumn{3}{c}{Avg IoU}\\
            Optimizer & Sequence length & Test & Validation & Train & Test & Validation & Train\\ \hline
            RMSProp & 3 & 78.0 & 90.2 & 91.5 & 37.8 & 58.4 & 60.3\\
            Adam & 3 & 78.8 & 90.3 & 91.4 & 38.2 & 58.3 & 59.8\\
            RMSProp & 1 & 84.5 & 94.2 & 96.5 & 50.7 & 72.1 & 84.7
        \end{tabular}
    }
    \caption{The results of the experiments on the effects of time correlation
        on the CamVid dataset. Pixel accuracy and average Intersection over
        Union (Avg. IoU) are reported. The model seems to fail to generalize
        properly.}
    \label{tbl:deconvLSTM_camvid_overfit}
\end{table}

% ADAM seq_len 3
% =============================
% test_acc: 0.788309417551
% test_jaccard: 0.38183944114
% valid_acc: 0.903313517338
% valid_jaccard: 0.582879679893
% train_acc: 0.914322370271
% train_jaccard: 0.597896683484

% RMSProp seq_len 3
% =============================
% test_acc: 0.780264451
% test_jaccard: 0.377975928479
% valid_acc: 0.901692251912
% valid_jaccard: 0.58374693901
% train_acc: 0.915286386415
% train_jaccard: 0.602986412809


%%%%%%%%% GATECH %%%%%%%%%
Most of the training effort was spent on the Gatech dataset, a well known
dataset for structured prediction. The performance of the DEConvLSTM model on
this dataset is remarkable, as reported in \autoref{tbl:deconvLSTM_gatech}. The
model aligns with the state-of-the-art without resorting to data augmentation
techniques, or finetuning on full resolution (i.e., without cropping the
frames), which can be expected to further improve the result, setting a new
state-of-the-art for this dataset. \autoref{fig:deconvlstm_gatech_samples}
presents some samples on the Gatech dataset. Although the model is usually able
to produce precise segmentation even on fine-grained details of the most
represented classes, such as sky and solid (the buildings), it can suffer on
the less frequent ones, such as humans and cars. A more in-depth discussion on
this issue will be provided in the following section.

The research carried out on this dataset allowed to evaluate the effectiveness
of the proposed architecture in non-trivial settings where the time component
has to be exploited to obtain convincing predictions. The results on the Gatech
dataset show that the proposed architecture can address successfully the video
segmentation task and suggest to carry on experimentation, e.g., adding data
augmentation and finetuning the network on full resolution sequences, to
further improve the state of the art.

\begin{table}[t]
    \centering{\small
        \begin{tabular}{l||c}
            Method & Global accuracy\\ \hline

            2D-V2V \citep{Tran16v2v} & 55.7 \\
            V2V-0 \citep{Tran16v2v} & 66.7 \\
            Conv3b + Up \citep{Tran16v2v} & 69.7 \\
            Conv4b + Up \citep{Tran16v2v} & 72.7 \\
            Conv5b + Up \citep{Tran16v2v} & 72.1 \\
            \textbf{V2V} \citep{Tran16v2v} & \textbf{76.0} \\
            \textbf{DEConvLSTM} & \textbf{76.1}
        \end{tabular}
    }
    \caption{Results on the Gatech dataset. Pixel accuracy is reported (higher
        is better). IoU is not reported as customary in the literature for this
        dataset.}
    \label{tbl:deconvLSTM_gatech}
\end{table}

\begin{figure}[p]
    \captionsetup[subfigure]{justification=centering}
    \begin{minipage}[c]{.95\textwidth}
        \begin{subfigure}{.47\textwidth}
            \centering
            \includegraphics[trim={13 10mm 25mm 10mm},clip,width=\linewidth]{img/deconvLSTM/gatech_10_204-0.png}
            \caption{}\label{fig:deconvlstm_gatech_samples:a}
        \end{subfigure}%
        \begin{subfigure}{.47\textwidth}
            \centering
            \includegraphics[trim={13 10mm 25mm 10mm},clip,width=\linewidth]{img/deconvLSTM/gatech_10_339-2.png}
            \caption{}\label{fig:deconvlstm_gatech_samples:b}
        \end{subfigure}
        \begin{subfigure}{.47\textwidth}
            \centering
            \includegraphics[trim={13 10mm 25mm 5mm},clip,width=\linewidth]{img/deconvLSTM/gatech_24_99-0.png}
            \caption{}\label{fig:deconvlstm_gatech_samples:c}
        \end{subfigure}
        \begin{subfigure}{.47\textwidth}
            \centering
            \includegraphics[trim={13 10mm 25mm 5mm},clip,width=\linewidth]{img/deconvLSTM/gatech_29_69-1.png}
            \caption{}\label{fig:deconvlstm_gatech_samples:d}
        \end{subfigure}
    \end{minipage}%
    \begin{minipage}[c]{.1\textwidth}
        \includegraphics[width=.7\textwidth]{img/deconvLSTM/gatech_barra.png}
    \end{minipage}%
    \caption{Segmentation samples on the Gatech dataset.}
	\label{fig:deconvlstm_gatech_samples}
\end{figure}


%%%%%% DAVIS %%%%%%%%%%
Encouraged by the good results on Gatech, we also tested the network on the
recently introduced DAVIS dataset for dense semantic segmentation. With only
cursory hyperparameter search around the settings that worked best on the
Gatech dataset, the DEConvLSTM obtains results comparable with the
state-of-the-art of a few years ago. \autoref{tbl:deconvLSTM_davis} reports the
results of the initial experiments with the DEConvLSTM model on this dataset
and \autoref{fig:deconvlstm_davis_sample} shows an example of prediction made
by the model.

The experiments conducted on this dataset are not enough to draw final
conclusions, but one failure mode of the DEConvLSTM model on this task seems to
be not to recognize the goal properly. The task in DAVIS is to segment only the
main element in the scene and by inspecting the segmentation samples (see,
e.g.,~\autoref{fig:deconvlstm_davis_failure}) it appears that sometimes
the network has a hard time identifying what to segment and what to ignore and
ends up focusing too much on the background, heavily decreasing the overall
segmentation performance. More experiments have to be conducted to address this
issue; one option could be to
% impose a stronger spatial prior with, e.g., a CRF, to focus more on the
% central part of the frame where the objects of interest are usually placed.
% Alternatively,
increase the capacity of the temporal pathway to rely more on the temporal
information to segment the main element of the scene that in these datasets is
usually the one that moves and changes the most. This early result is very
encouraging and calls for more thorough experiments on this dataset.

\afterpage{
\begin{table}[t]
    \centering{\small
        \begin{tabular}{l||c}
            Method & Avg IoU\\ \hline

            Temporal superpixels \citep{chang2013video} & 35.8 \\
            SeamSeg \citep{ramakanth2014seamseg} & 55.6 \\
            Efficient hierarchical graph-based video segmentation \citep{
                grundmann2010efficient} & 59.6 \\
            Jumpcut \citep{fan2015jumpcut} & 60.7 \\
            Fully Connected Object Proposals for Video Segmentation \citep{
                perazzi2015fully} & 63.1 \\
            \textbf{Bilateral space video segmentation} \citep{
            marki2016bilateral} & \textbf{66.5} \\
                \emph{DEConvLSTM} & 56.8
        \end{tabular}
    }
    \caption{Results on the Davis dataset. Average Intersection over Union
        (IoU) is reported. We compare to~\protect\footnotemark, where pixel accuracy is
        not reported.}
    \label{tbl:deconvLSTM_davis}
\end{table}
\footnotetext{\url{https://graphics.ethz.ch/~perazzif/davis/soa_compare.php}}
}

\begin{figure}[p]
    \centering
    \includegraphics[trim={0 15mm 0 0},width=0.9\columnwidth]{img/deconvLSTM/paragliding-0.png}\\
    \caption{An example of proper segmentation produced by the DEConvLSTM model
        on the DAVIS dataset.}
    \label{fig:deconvlstm_davis_sample}
\end{figure}

\begin{figure}[t]
    \centering
    \includegraphics[trim={0 15mm 0 0},width=0.8\columnwidth]{img/deconvLSTM/drift-0.png}\\
    \includegraphics[trim={0 15mm 0 0},width=0.8\columnwidth]{img/deconvLSTM/bmx-trees-0.png}
    \caption{A frequent failure mode of the model on the DAVIS dataset is to
        focus on segmenting the background rather than the main element in the
        scene.}
    \label{fig:deconvlstm_davis_failure}
\end{figure}


\section{Discussion}\label{sec:deconvLSTM_discussion}
One of the first goals of the experiments with the DEConvLSTM model was to
assess its effectiveness in exploiting the temporal correlation among frames in
the task of video spatial segmentation. To this end, several tests were run on
CamVid, training the same model with different lengths of the sequence.

CamVid has been mostly used in the literature in a single image
segmentation setting, rather than considering the frames altogether as a video.
Indeed, on this dataset the model performed better when the sequence length was
set to 1, i.e., when the temporal correlation was not taken into account. This
is not too surprising, since this dataset is composed by frames captured at 30
fps, but only one frame out of approximately 30 is annotated and available for
training. The experiments show that trying to exploit temporal correlation at 1
Hz is very non-trivial, and results in a massive drop of performance as
emerges clearly from the results reported
in~\autoref{tbl:deconvLSTM_camvid_overfit}. Another aspect that clearly emerges
from the results is the performance drop between the test set and the other
sets. Overfitting is an important problem when training on the CamVid dataset,
due to the size of the training set that is quite small, especially for big
models like the DEConvLSTM.

On the contrary, there is evidence in the literature that exploiting the
temporal correlation between frames improves performance on the Gatech
dataset~\citep{Tran16v2v}. The results reported in \autoref{%
tbl:deconvLSTM_gatech} show that the DEConvLSTM model is an interesting
architecture for the video semantic segmentation task. Furthermore,
generalization seems to be much better on Gatech with respect to CamVid, in
fact the drop in performance between the validation and the test set is
remarkably contained around $2\%$, as opposed to the $10-15\%$ drop typically
experienced in CamVid.

By analyzing some of the segmentation samples (see
\autoref{fig:deconvlstm_gatech_samples}) it is evident that the model learns a
good representation of the most represented objects of the dataset (e.g., the
buildings (labeled as solid) and the sky), and is able to segment fine-grained
details such as the leaves of the trees
(\autoref{fig:deconvlstm_gatech_samples:a}) and, although not perfectly, far
away people (such as those in ~\autoref{fig:deconvlstm_gatech_samples:d}).
There are still some imperfections in the segmentation which could be caused by
misleading ground truth labels that could be making it less clear for the model
to distinguish between some of the classes. Take for
example~\Autoref{fig:deconvlstm_gatech_samples:a,
fig:deconvlstm_gatech_samples:b}, where the ground truth label for some of the
trees is "porous" and for some other is "vertical mix"; similarly in
\autoref{fig:deconvlstm_gatech_samples:c} part of the street is labeled as
"ground" and some of it as "general mix", even if there is no apparent
difference between the two parts of the street in the image. These kind of
unclear semantic is a probable cause for the uncertainty of the model in some
cases, such as the mixed segmentation of the building in the background
in~\autoref{fig:deconvlstm_gatech_samples:d}.

Unfortunately the model did not perform as well when trained \emph{using the
same hyperparameters} on the Davis dataset. The results on this dataset are not
conclusive as the dataset is fairly new and not enough experimentation has been
conducted to draw any conclusion. It seems that the model could benefit from an
exploration of the hyperparameters that best push it to focus more on the main
object of the scene. This could of course also be obtained by more invasive
modifications of the model, e.g., the addition of an attention mechanism or of
a spatial prior that could be learned by a CRF, or in a more trivial way just
by adding a hand-engineered spatial prior, as the object of interest tends to
be centered in the frames. Nevertheless, since our interest lies more in
assessing the learning capabilities of the model rather than in obtaining the
absolute performance value, we plan to focus mostly on the study of variations
of the current architecture rather than on relying on hand engineered solutions
to improve the final score.

One interesting discovery, valid for all the experiments ran so far, is that
using the per-batch statistics for batch normalization at test time (rather
than the more common strategy of re-using the running statistics collected
during training) significantly improved the results on all the three datasets.
In this setting the statistics to perform the feature-wise normalization of the
activations of the network are computed on the test minibatch under exam,
as opposed to using the statistics collected on the whole training set. This
seems to suggest a difference in the distributions of the data in the training
and the test set or, more likely, a higher-than-usual variance of the images
of the dataset. Bigger dataset would likely not suffer from this issues.

Another crucial factor to achieve a good optimization is to select the right
amount of momentum in the computation of the batch normalization statistics. We
argue that the reason for both behaviors lies in the high variance of the input
data. The datasets are in fact composed by sequences of frames coming from
different videos that can vary consistently in terms of, e.g., brightness,
scale and noise. Relying too heavily on past statistics in this setting can be
harmful.

Video semantic segmentation is a hard task to tackle. CNNs are known to be
data hungry, as proved by the great performance leap that followed the
introduction of large datasets for object classification~\citep{ILSVRCarxiv14}.
The next milestone for computer vision research is reaching a good
understanding of videos in structured prediction related tasks. Good datasets
with large amounts of densely annotated video data are still missing, even if
the community is beginning to make an effort to fill in the gap~\citep[%
see~e.g.,~][]{Perazzi2016,lin2014microsoft}. More work is needed to speed up
the models, study new loss functions that allow to optimize more closely
non-derivable metrics such as IoU and reuse computation exploiting similarity
in consecutive frames. Nonetheless it is already possible to challenge
non-neural methods and reach reasonable results - when not state-of-the-art
performance.  Indeed the results obtained with DEConvLSTM show that coupling
CNNs, transposed CNNs and LSTMs can be an effective way to tackle structured
prediction problems in video, building on the speed of CNNs to process spatial
information and on the ability of RNNs to retain information through several
steps of computation.

%to handle variable size images we could just have used something like SPP
% the argument is that TIME really has to make a difference
%FCN8, FCN8 with convLSTM, our model

%/data/lisa/exp/romerosa/deconvRNN/tmp/
%/data/lisatmp4/romerosa/deconvRNN
